% !TEX root=exama-221020.tex
%--------------------------------------
% Create title frame
\titleframe

%--------------------------------------
% Table of contents
\begin{frame}{Overview}
  \setbeamertemplate{section in toc}[sections numbered]
  \tableofcontents[hideallsubsections]
\end{frame}


%==============================================
\section{ExaMA: Methods and Algorithms at Exascale}
%==============================================
%==============================================
\subsection{NumPEx::PC1}
\begin{frame}{\insertsectionhead}
  \framesubtitle{ExaMA $\equiv$ PC1 $\equiv$ IP1}
  NUMPEX/ExaMa concentrates on the exascale aspects of the numerical methods, ensuring their scalability to existing and forthcoming hardware.
  \vfill
  Leaders: C Prud'homme \& H Barucq
  \begin{itemize}
    \item 5 Work packages
    \item wide range of topics: 
    \begin{itemize}
        \item Modeling and discretize
        \item Linear, multi-linear and coupled solvers at Exascale
        \item Combine data and  models at Exascale
        \item Optimize and quantify uncertainties at Exascale
    \end{itemize}
    \item Demonstrators through mini-apps will be used to verify the properties of the methods and algorithms developed.
  \end{itemize}
\end{frame}

\subsection{NumPEx::PC1 Team (a work still in progress)}

\begin{frame}
  \frametitle{\insertsectionhead}
  \framesubtitle{\insertsubsectionhead}

  Organismes financés
  \begin{itemize}
    \item CEA : DES(1 - 2), DAM (1)
    \item INRIA : Bordeaux(2-4),  Côte d'Azur (2), Grenoble(1), Lille(1), Paris(1)
    \item IPP (CMAP, Inria ASCII, Inria POEMS)
    \item UNISTRA  (IRMA-MOCO/Cemosis, Inria Tonus)
  \end{itemize}

  \begin{alertblock}{Other teams}
    \begin{itemize}
      \item Sorbonne Université ? (LJLL: Y  Maday, S Labbé; LIP6 Theo Marie, P Jolivet)
      \item ENS Lyon ? (Y Robert, ....) 
    \end{itemize}
  \end{alertblock}
\end{frame}

\subsection{Budget}

\begin{frame}
  \frametitle{\insertsectionhead}
  \framesubtitle{\insertsubsectionhead}
  \begin{itemize}
    \item si un universitaire est dans une equipe inria financée par Numpex, pas de souci
    \item Sinon comment collaborer (si necessaire)? Sur les sites, pourra t'on nous appuyer sur les conventions existantes pour 
    mettre en place des collaborations/des co-financements/..../co-directions/... ?
    \begin{itemize}
      \item Bordeaux: 
      \item Cotes d'Azur : 
      \item Grenoble : 
      \item Lille:  : 
      \item Paris : 
      \item Saclay : 
      \item Strasbourg : 
    \end{itemize}
  \end{itemize}
  \alert{Recemment accord cadre inria / cnrs signé}

\end{frame}

\section{Steering team}
\begin{frame}
  \frametitle{\insertsectionhead}
  \framesubtitle{\insertsubsectionhead}
\scriptsize
  \begin{itemize}
    \item CEA 
    \begin{itemize}
      \item DAM \textbf{Lydie Grospellier} (LGr)
      \item DES \textbf{Vincent Faucher} (VF) \textbf{Isabelle Ramière} (IR)  
    \end{itemize}
    \item INRIA 
    \begin{itemize}
      \item Bordeaux \textbf{Hélène Barucq} (HB) \textbf{Luc Giraud} (LGi)
      \item  Grenoble \textbf{Arthur Vidard} (AV)
      \item Lille \textbf{El-Ghazali Talbi} (ET) 
      \item Paris \textbf{Laura Grigori} (LG) \textbf{Frédéric Nataf} (FN)
      \item Sofia \textbf{Stephane Lanteri}(INRIA-Sofia) (SL) 
    \end{itemize}
    \item IPP \textbf{Josselin Garnier} \textbf{Marc Massot} (MM) \textbf{Loic Gouarin} (LGo)
    \item UPICARDIE \textbf{Mark Asch} (MA)
    \item UNISTRA \textbf{Christophe Prud'homme}(CP) \textbf{Emmanuel Franck} (EF) \textbf{Yannick Privat} (YP)
  \end{itemize}
  \alert{to be completed}
\end{frame}

\section{Identified Bottlenecks/Challenges}
\subsection{Challenges}
\begin{frame}
  \frametitle{\insertsectionhead}
  \framesubtitle{\insertsubsectionhead}
  \footnotesize
  \begin{columns}
    \column{.5\textwidth}
    \begin{itemize}
      \item (C1) Reduce carbon (GHG) footprint in transportation, buildings, and cities
       \item (C2) Design, control, and manufacture of advanced materials
       \item (C3) Understand and simulate the human brain
       \item (C4) Understand fission and fusion reactions and design advanced experiment facilities for fusion
             \end{itemize} 
    \column{.5\textwidth}
    \begin{itemize}
      \item (C5) Monitor the health of our planet: climate prediction, impact assessment of environmental policies, rapid environmental hazards

        \item (C6) Monitor and personalize the health of human beings 
       \item (C7) Design drugs
       \item (C8) Design cost-effective renewable energy resources: batteries, biofuels, solar photovoltaics
       \item (C9) Understand the Universe
     \end{itemize} 
  \end{columns}

\end{frame}
\subsection{Bottlenecks}
\begin{frame}[fragile=singleslide]{\insertsectionhead}
  \framesubtitle{\insertsubsectionhead}
  \footnotesize
  \begin{columns}[]
    \begin{column}{.5\linewidth}
      \begin{itemize}
        \item (B1) Energy efficiency
        \item (B2) Interconnect Technology
        \item (B3) Memory technology
        \item (B4) Scalable systems software
        \item (B5) Programming systems
        \item (B6) Data Management
        \item (B7) Exascale Algorithms
      \end{itemize}
    \end{column}
    \begin{column}{.5\linewidth}
      \begin{itemize}
        \item (B8) Discovery, design, and decision algorithms
        \item (B9) Resilience, robustness and accuracy
        \item (B10) Scientific productivity
        \item (B11) Reproducibility, replicability of computation
        \item (B12) Pre/Post-processing
        \item (B13) Integrate Uncertainties
      \end{itemize}
    \end{column}
  \end{columns}

  
\end{frame}
\section{Work Packages}
\subsection{WP1: Modeling and Discretization}
\begin{frame}
  \frametitle{\insertsectionhead}
  \framesubtitle{\insertsubsectionhead}

  \begin{columns}
    \column{.5\textwidth}
    \begin{itemize}
      \item Geometric representation and their discrete counterparts [B2, B6, B7, B9, B11-B13] 
      \item physics-based models[B7, B10] 
    \end{itemize}
    \column{.5\textwidth}
  
  \begin{alertblock}{Data }
    Contributors VF, MM, PA, CP, PH
    Links with PC2-WP2/3, PC3-WP3 

  \end{alertblock}

  \end{columns}
\end{frame}


\subsection{WP2: Reduced order and AI driven methods for multi-fidelity modeling} 
\begin{frame}
  \frametitle{\insertsectionhead}
  \framesubtitle{\insertsubsectionhead}

  \begin{columns}
    \column{.5\textwidth}
    \begin{itemize}
      \item AI-driven, data-driven, reduced-order, and more generally surrogate models[B2, B7, B8, B10-B13]
      \item Multi-fidelity models [B2, B7, B8]
    \end{itemize}
    \column{.5\textwidth}
  
    \begin{alertblock}{Data}
    Contributors: SL, EF, HB, CP, JG\\
    Links with PC2-WP2/3, PC3-WP3
  \end{alertblock}

  \end{columns}
\end{frame}

\subsection{WP3: Linear, Multi-linear and Coupled Solvers at Exascale}
\begin{frame}
  \frametitle{\insertsectionhead}
  \framesubtitle{\insertsubsectionhead}
  \begin{columns}
    \column{.5\textwidth}
    \begin{itemize}
      \item Acceleration techniques for subspace-based methods [B1, B2, B5, B7, B9-B10].
      \item High dimensional problems [B1, B2, B5, B7, B10] 
      \item Randomization [B1, B2, B7, B10]
      \item Exploiting data-sparsity and multiple precision [B1, B2, B5, B7, B10]
      \item Adaptive solution strategies for exascale multiphysical and multiscale models [B7, B9-B11] 
    \end{itemize}
    \column{.5\textwidth}

    \begin{alertblock}{Data}
    Contributors: LG, LGi, VF, FN, PJ, ...
    Links with PC2-WP2/3   
  \end{alertblock}
  \end{columns}

\end{frame}

\subsection{WP4: Combine data and models, inverse problems at Exascale }
\begin{frame}
  \frametitle{\insertsectionhead}
  \framesubtitle{\insertsubsectionhead}
  \begin{columns}
    \column{.5\textwidth}
    [B2, B6, B7, B8, B13]
    \begin{itemize}
      \item Deterministic methods
      \item Stochastic methods
      \item Observations
      \item Taking advantage of multi-fidelity modeling
      \item challenges of multi-fidelity in inverse problems: criteria to update reduced models
    \end{itemize}
    \column{.5\textwidth}
    \begin{alertblock}{Data }
    Contributors: AV, MA, HB, CP, JG\\
    Links with PC2-WP2/3,PC3-WP3
  \end{alertblock}
  \end{columns}
\end{frame}

\subsection{WP5: Optimize at Exascale }
\begin{frame}
  \frametitle{\insertsectionhead}
  \framesubtitle{\insertsubsectionhead}
  \begin{columns}
    \column{.5\textwidth}
    [B6-B8, B10, B13]
    \begin{itemize}
      \item Optimization 
      \begin{itemize}
        \item shape, dynamic shape optimization
        \item combinatorial optimization
        \item policy based optimization
        \item automated learning/AI for advanced design
      \end{itemize}
    \end{itemize}
    \column{.5\textwidth}
    \begin{alertblock}{data}
      Contributors: ET, YP, CP\\
      Links with PC2-WP2/3,PC3-WP2 
    \end{alertblock}
  \end{columns}
\end{frame}

\subsection{WP6: Quantify uncertainty at Exascale - Links with P2-WP2/3,P3-WP2/3 }
\begin{frame}
  \frametitle{\insertsectionhead}
  \framesubtitle{\insertsubsectionhead}
  \begin{columns}
    \column{.5\textwidth}
    [B6-B8, B10, B13]
    \begin{itemize}
      \item Uncertainty quantification including 
      \begin{itemize}
        \item uncertainty propagation
        \item sensitivity analysis
        \item robust inversion
        \item UQ at different scales
        \item weak vs strong UQ
      \end{itemize}
    \end{itemize}
    \column{.5\textwidth}
    \begin{alertblock}{data}
      Contributors: JG, (JMM,) MA\\
      Links with PC2-WP2/3,PC3-WP2/3
    \end{alertblock}
  \end{columns}
\end{frame}

\subsection{WP7: Demonstrate methods and algorithms at Exascale}
\begin{frame}
  \frametitle{\insertsectionhead}
  \framesubtitle{\insertsubsectionhead}

  \begin{columns}
    \column{.5\textwidth}
    [B1-B13]
    \begin{itemize}
      \item Properties Verification on small/mini apps within PC1
      \item Co-design with the CDT and PC5
    \end{itemize}
    \column{.5\textwidth}
    \begin{alertblock}{Data }
      Contributors: LGr et ALL\\
      Links with PC2-WP2/3,PC3-WP2/3 and PC5
    \end{alertblock}
   
  \end{columns}
\end{frame}

\subsection{Deliverables}

\begin{frame}
  \frametitle{\insertsectionhead}
  \framesubtitle{\insertsubsectionhead}
  \begin{itemize}
      \item   Methods, algorithms, and implementations that, taking advantage of the exascale architectures, empower modeling, solving, assimilating model and data, optimizing and quantifying uncertainty, at levels that are unreachable at present.
    \item Software libraries allowing to assemble specific critical reusable components, hiding the hardware complexity and exposing only the specific methodological interface
    \item Methodological and Algorithmic Patterns at exascale that can be reused efficiently in large scale applications (eg in weather forecasting)
    \item Enabling AI algorithms to attain performances at exascale, exploiting the methods (point 1) and the libraries (point 2) developed.
    \item \href{https://docs.google.com/document/d/1hjwSFRF63SyTUJJKGMNLHcJPr_S2JDHYXeBeQzHCSno/edit?usp=sharing}{\beamergotobutton{Demonstrators}}

  \end{itemize}


\end{frame}

\subsection{Milestones}

\begin{frame}
  \frametitle{\insertsectionhead}
  \framesubtitle{\insertsubsectionhead}

  \begin{itemize}
    \item   M1 Select IP-1 use-cases/demonstrators and associate methodology developments T0+6
    \item M2 benchmark IP-1 demonstrators on pre-exascale systems T0+9/T0+12
    \item M3 enable and benchmarks some new exascale  IP-1 components on pre-exascale/exascale systems T0+18, T0+36, T0+54, T0+60
\end{itemize}


\end{frame}




\section{Relations}
\subsection{Entreprises}
\begin{frame}
  \frametitle{\insertsectionhead}
  \framesubtitle{\insertsubsectionhead}
  \begin{alertblock}{Entreprises}
    \begin{itemize}
      \item Will depend on final team, will be discussed in next coordination meeting
      \item Expected: EDF, Safran, Total, Atos
      \item Others: PlasticOmnium, Arkema, Entreprise consortium MOR\_DICUS...
    \end{itemize}
  \end{alertblock}
\end{frame}

\subsection{EPIC \& PEPR}
\begin{frame}
  \frametitle{\insertsectionhead}
  \framesubtitle{\insertsubsectionhead}
\begin{alertblock}{EPIC}
  \begin{itemize}
    \item  Will depend on final team, will be discussed in next coordination meeting
    \item Expected: Onera(discussions also next week)
  \end{itemize}
\end{alertblock}

\begin{alertblock}{PEPR}
  \begin{itemize}
    \item Expexted: IA
    \item Others: Diadem, TRACCS-Météo...
  \end{itemize}
\end{alertblock}

\end{frame}

\subsection{Europe}
\begin{frame}
  \frametitle{\insertsectionhead}
  \framesubtitle{\insertsubsectionhead}
  \begin{alertblock}{CoE}
    \begin{itemize}
      \item Will depend on final team, will be discussed in next coordination meeting
      \item Expected: Hidalgo2, Cheese
      \item Others: CoE	EoCoE-3
    \end{itemize}
  \end{alertblock}

  \begin{alertblock}{Europe}
    \begin{itemize}
      \item Will depend on final team, will be discussed in next coordination meeting
      \item Others: ERC-Synergy	EMC2, EuroHPC	Microcard, H2020 RIA Digital Twin	Bim2Twin,EuroHPC	European Master for HPC - EUMaster4HPC	
    \end{itemize}
  \end{alertblock}
\end{frame}

\section{Project Management}

\subsection{Principles}
\begin{frame}[fragile=singleslide]{\insertsectionhead}
  \framesubtitle{\insertsubsectionhead}

  \begin{itemize}
    \item \textbf{Openness} and \textbf{transparency} of the project 
    \item \textbf{Collaboration} with other projects : 
    \begin{itemize}
      \item 
        co-design with PC5, collaboration with PC2,3,4\
        \item 
          collaboration with other projects e.g. EuroHPC projects(Coe) and other PEPR (IA, Diademe,TRACCS-Météo...
    \end{itemize}
    \item \textbf{Inclusiveness} of the community 
    \begin{itemize}
      \item use the project as leverage for co-funding  or, also, collaborating outside the project eg phd co-advisors
      \item training : initial(train future PhD students) and continuous (broader community)
    \end{itemize}      
  \end{itemize}

\end{frame}

\subsection{Work plan}
\begin{frame}
  \frametitle{\insertsectionhead}
  \framesubtitle{\insertsubsectionhead}
  \begin{alertblock}{Project Management}
    \begin{itemize}
      \item Several co-leads per WP 
      \item Meeting almost every week to advance the writing

    \end{itemize}
  \end{alertblock}

\begin{alertblock}{Tools}
  \begin{itemize}
    \item  Use of Google Doc and GitHub (repo and project management)
    \item Creation  of an archived mailing list 
  \end{itemize}
\end{alertblock}
\end{frame}

\section{Budget}

\begin{frame}
  \frametitle{\insertsectionhead}

  
  

\end{frame}
\section{Questions}

\begin{frame}
  \frametitle{\insertsectionhead}  

  \begin{itemize}
    \item Quid des thématiques transverses ? eg résilience, energie 
    \item Quelle est la stratégie logicielle ? Open-Source ? Closed Source ? Probablement 80/90\% open source, le reste fermé
    \item doit-on déclarer en partenaire un établissement parce qu'un de ces membres participe au projet? C'est plutôt oui même si la personne n'intervient pas avec un financement de Numpex. A voir donc avec le consortium.
    \item doit-on prévoir les possibles co-financements externes dans le montage du projet? autrement dit, les accords existant déjà entre les différents établissements sont-ils suffisants pour mettre en place des co-financements? Il existe des ED qui refusent qu'il y ait deux financeurs de thèse différents, besoin de conventions de reversement donc.
  \end{itemize}

  

\end{frame}